\documentclass[11pt,a4paper]{article}

%----------------------------
% Encoding & Language
%----------------------------
\usepackage[utf8]{inputenc}
\usepackage[T1]{fontenc}
\usepackage[english]{babel}

%----------------------------
% Page Layout
%----------------------------
\usepackage[
  a4paper,
  left=25mm,
  right=25mm,
  top=25mm,
  bottom=30mm,
]{geometry}

%----------------------------
% Mathematics & Physics
%----------------------------
\usepackage{amsmath}       % improved math environments
\usepackage{amssymb}       % extra math symbols
\usepackage{mathtools}     % fixes and extensions to amsmath
\usepackage{physics}       % physics notation (\abs, \dv, etc.)

%----------------------------
% Units & Numbers
%----------------------------
\usepackage{siunitx}       % SI units and number formatting
\sisetup{
  detect-all,
  separate-uncertainty = true,
  per-mode = symbol
}

%----------------------------
% Graphics & Figures
%----------------------------
\usepackage{graphicx}      % includegraphics
\usepackage{xcolor}        % color support
\usepackage{float}         % [H] placement specifier
\usepackage[font=small,labelfont=bf]{caption}
\usepackage[subrefformat=parens]{subcaption}  % subfigures

%----------------------------
% Tables
%----------------------------
\usepackage{booktabs}      % \toprule, \midrule, \bottomrule
\usepackage{multirow}
\usepackage{array}

%----------------------------
% Bibliography & Hyperlinks
%----------------------------
\usepackage[
  backend=biber,
  style=apa,
  natbib=true
]{biblatex}
\addbibresource{references.bib}

\usepackage[
  hidelinks,
  pdfauthor={Elie Habib},
  pdftitle={Astrophysics Lab Report},
]{hyperref}

%----------------------------
% Miscellaneous
%----------------------------
\usepackage{microtype}     % typographical refinements
\usepackage{enumitem}      % control over list spacing
\usepackage{datetime2}     % date formatting
\usepackage{titlesec}      % control of section headings

%----------------------------
% Title Page Definition
%----------------------------

\newcommand{\reporttitle}{Astrophysics Lab Report}
\newcommand{\authorname}{Elie Habib and Jasmin Bsul}
\newcommand{\instructor}{Sara Faris}
\newcommand{\coursename}{Astrophysics Lab}

%----------------------------
% Document
%----------------------------
\begin{document}

% Title page
\begin{titlepage}
  \centering
  \vspace*{1cm}
  {\Huge\bfseries \reporttitle \par}
  \vspace{1.5cm}
  {\Large\coursename \par}
  \vspace{2cm}
  {\Large\authorname \par}
  \vfill
  Instructor: \instructor\par
\end{titlepage}

% Table of Contents
\cleardoublepage
\pagenumbering{roman}
\tableofcontents
\cleardoublepage
\pagenumbering{arabic}
%============================

\section{Introduction}
\subsection{Theory}
Galactic analysis involves examining electromagnetic radiation emitted by galaxies across various wavelengths in order to infer critical physical properties. 
Four "pictures" is taken using the huble Telescope in  the Hubble deep field where a large number of galaxies can be seen. Where each one of the pictures is taken using different sensors that are sensitive to different wavelengths of light. 

the magnitudes of the light captured on the sensors can help us learn about the different properties of the different galaxies. 


Galaxies are typically classified into morphological categories including elliptical and spiral. Elliptical galaxies generally contain older stellar populations which are characterized by cooler, lower-mass stars, resulting in redder observed colors. In contrast, spiral galaxies possess significantly younger stars that are hotter, and more massive and immit bluer radiation, or even ongoing star formation activities, particularly within their spiral arms.

The color index, defined as the observed color of the galaxy, and it is calculaetd by taking the difference in magnitude between observations at two different wavelengths,  it provides a quantitative measure of galaxy colors. This index helps us distinguish between galaxy types, as galaxies with active star formation tend to show bluer (lower color index) values, and galaxies with older star populations appear redder (higher color index).

Redshift ($z$), is particularly important in analyzing distant galaxies. Redshift arises due to the expansion of the universe, which stretches the wavelengths of electromagnetic radiation emitted by distant sources as they travel toward the observer. The redshift ($z$) is defined mathematically as:

\begin{equation}
 z =  \frac{\lambda - \lambda_0}{\lambda_0}
\end{equation}



where $\lambda$ represents the observed wavelength and $\lambda_0$ the  wavelength measured in laboratory conditions. Redshift measurements help us estimate the distance and velocities of galaxies through Hubble's law, which relates galaxy velocity ($v$) to distance ($r$):

\begin{equation}
  v = H_0 \cdot r
\end{equation}


Here, $H_0$ denotes the Hubble constant.

In this la, digital images obtained the Hubble Space Telescopes in multiple photometric filters (U, B, V, I) allow for precise measurements of these properties. By comparing observed fluxes across these filters, a color index and approximate SED for each galaxy can be determined. comparing the measured SED and redshift with theoretical galaxy spectra enables classification of galaxy types and estimation of redshifts.


\subsection{reasearch question}
In this expirement we are interested in finding the corilation between the color index of a galaxy to its redshift value.


\pagebreak
\section{Data analysis process}
Since the data (pictures) were previusly taken using the hubble telescope all that is needed to


\end{document}